\documentclass[12pt]{article} % use larger type; default would be 10pt


%%% PAGE DIMENSIONS
\usepackage[lmargin=.5in, rmargin=.5in, marginpar=1.3in]{geometry} % to change the page dimensions (margins, etc.)
\geometry{letterpaper} % or letterpaper (US) or a5paper or....

\usepackage{graphicx} % support the \includegraphics command and options

% \usepackage[parfill]{parskip} % Activate to begin paragraphs with an empty line rather than an indent

%%% PACKAGES
\usepackage{amsmath}
\usepackage{upgreek}
\usepackage{dsfont}
\usepackage{setspace} % for changing line spacing
\usepackage{array} % for better arrays (eg matrices) in maths
\usepackage{paralist} % very flexible & customisable lists (eg. enumerate/itemize, etc.)
\usepackage{verbatim} % adds environment for commenting out blocks of text & for better verbatim
\usepackage{subfig} % make it possible to include more than one captioned figure/table in a single float
\usepackage{amsthm} % allows example and proof environment to be defined/used
\usepackage{amssymb}
\usepackage{todonotes}
\usepackage{pifont}
%%% HEADERS & FOOTERS
\usepackage{fancyhdr} % This should be set AFTER setting up the page geometry
\pagestyle{fancy} % options: empty , plain , fancy
\renewcommand{\headrulewidth}{2pt} % customise the layout...
\lhead{Algebraic Topology Questions}\chead{}\rhead{Graham Lewis}
\lfoot{}\cfoot{\thepage}\rfoot{}

%%% COMPLETE AND INCOMPLETE COMMANDS
\newcommand{\incomplete}{$\square$}
\newcommand{\complete}{$\boxtimes$}
\usepackage{ marvosym }
\newcommand{\graded}{\Smiley}
\newcommand{\Z}{\mathds{Z}}
\newcommand{\dist}{dist}
\newcommand{\boxes}{Box}
%%% DEFINE EXAMPLE AND THEOREM ENVIRONMENTS
\theoremstyle{definition}
\newtheorem{example}{Example}[section]
\newtheorem{theorem}{Theorem}


\begin{document}

\section*{Continuity}
\begin{enumerate}
    \item[Metric Equivalence:]  Assume $(X,d_{X})$ and $(Y,d_{Y})$ are metric spaces.
Show that a map $f:X\rightarrow Y$ is continuous at $x\in X$ if and only if for each $\varepsilon>0$, we can find a $\delta>0$ such that $d_{X}(x',x)<\delta$ for $x'\in X$ implies $d_{Y}(f(x'),f(x))<\varepsilon$.
\vspace{5mm}
\\
$(\Rightarrow)$ Let $x\in X$, $\varepsilon>0$.
Suppose $f:X\rightarrow Y$ is continuous.
Then we know $B(f(x),\varepsilon)$ is open so $f^{-1}(B(f(x),\varepsilon)$ is open.
Since $x\in f^{-1}(B(f(x),\varepsilon)$ there exists $\delta>0$ such that $B(x,\delta)\subseteq f^{-1}(B(f(x),\varepsilon)$.
Thus, we have $f(B(x,\delta))\subseteq B(f(x),\varepsilon)$ and thus if $d(x,y)<\delta$ then we have $d(f(x),f(y))<\varepsilon$.\\
$(\Leftarrow)$ Suppose for each $\varepsilon>0$ we can find a $\delta>0$ such that $d_{X}(x',x)<\delta$ for $x'\in X$ implies $d_{Y}(f(x'),f(x))<\varepsilon$.
Let $x\in X$ and consider $V(f(x))$.
Let $\varepsilon>0$ be such that $B(f(x),\varepsilon)\subseteq V(x)$.
Then we know that there exists $\delta>0$ such that if $y\in B(x,\delta)$ then $f(y)\in B(f(x),\varepsilon)$.
Thus, $B(x,\delta)\subset f^{-1}(B(f(x),\varepsilon))\subset f^{-1}(V(f(x)))$.
Then for every neighborhood $V(f(x))$ we know there exists a neighborhood $V(x)\subseteq f^{-1}(V(f(x)))$ so $f$ is continuous.
\end{enumerate}



\end{document}
